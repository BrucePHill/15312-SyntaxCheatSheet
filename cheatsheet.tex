% --------------------------------------------------------------
% This is all preamble stuff that you don't have to worry about.
% Head down to where it says "Start here"
% --------------------------------------------------------------
 
\documentclass[12pt]{article}
 
\usepackage[margin=1in]{geometry} 
\usepackage{amsmath,amsthm,amssymb}
\usepackage{fullpage}
\usepackage{times}
\usepackage{amsmath,proof,amsthm,amssymb,color}
\usepackage{ifthen}
\usepackage{hyperref}
 
\newcommand{\N}{\mathbb{N}}
\newcommand{\Z}{\mathbb{Z}}
 
\begin{document}
 
\title{Syntax Cheat Sheet}
 
\maketitle

\renewcommand{\arraystretch}{2.0}

\begin{tabular}{l | p{6cm} | p{6cm} }
	\bf{Syntax} & \bf{English} & \bf{Example} \\
	\hline & & \\

	$\infer{c}{p_1 \; p_2 \; ... \; p_n}$ & $c$ can be concluded from premises $p_1$, $p_2$, ..., and $p_n$ & $\infer{plus(a,b) \; nat}{a\; nat \quad b \; nat}$\\

	$a \mapsto b$ & $a$ evaluates to $b$ &  N/A\\

	$[a/x]e$ & Substitute $a$ for $x$ in the expression $e$ & $[4/x]times(x; y) \mapsto times(4; y)$\\

	$\Gamma$ & A set of rules & N/A \\

	$\Gamma \vdash K$ & $K$ is logically derivable from $\Gamma$ & $\infer{\Gamma \vdash succ(x) \; nat}{\Gamma \vdash x \; nat}$ \\
	$e : \tau$ & $e$ has type $\tau$ & $x : nat$ \\

	$e \Downarrow v$ & $e$ has value $v$ & $x \Downarrow 4$ \\

	$x.e$ & The variable $x$ in the expression $e$ & $x.times(x;y)$ \\

	let($a_1$; $x.a_2$) & Let $x$ be $a_1$ in $a_2$ & $let(x; 4 . times(x; y)) \mapsto times(4; y)$ \\

	ar($o$) = $(s_1, s_2, ..., s_n)$ & Operator $o$ has arity $(s_1, s_2, ..., s_n)$ (i.e. it has $n$ arguments that have sorts $s_1, s_2, ..., s_n$ respectively) & ar($let$) = (Exp, (Exp)Exp) \\


\end{tabular}

\end{document}